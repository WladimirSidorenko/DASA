% FILE: dasa.tex  Version 0.0

%% Imports
\documentclass[11pt]{article}
\usepackage{coling2020/coling2020}
\usepackage{times}

\usepackage{amsmath}
\usepackage{array}
\usepackage{booktabs}
\usepackage{multicol}           % for multiple columns in a table
\usepackage{multirow}
\usepackage{paralist}
\usepackage{rst}
\usepackage[colorinlistoftodos,prependcaption,textsize=tiny]{todonotes}

%% Commands and Macros
\newcommand{\F}[0]{$F_1$}
\newcommand{\argmax}{\operatornamewithlimits{argmax}}
\newcommand{\argmin}{\operatornamewithlimits{argmin}}
\newcommand{\eg}{\textit{e.g.},}
\newcommand{\ienocomma}{\textit{i.e.}}
\newcommand{\ie}{\ienocomma,}
\newcommand{\discrel}[1]{\texttt{#1}}

%% Main
\title{Discourse-aware Sentiment Analysis Using Recursive Dirichlet
  Model}

\author{Anonymous Author\\
  Affiliation line 1 \\
  Affiliation line 2 \\
  Affiliation line 3 \\
  {\tt email@example.com} \\\And{}
  Anonymous Author\\
  Affiliation line 1 \\
  Affiliation line 2 \\
  Affiliation line 3 \\
  {\tt email@example.com} \\}

\date{}

\begin{document}
\maketitle
\begin{abstract}
  This paper presents a novel Bayesian approach to sentiment analysis
  of texts, in which we estimate the overall polarity of a discourse
  from sentiment scores of its elementary discourse units (EDUs) using
  a dedicated probabilistic magma.  In particular, we regard the
  polarity of each text unit as a multinomial random variable, whose
  conjugate prior is a Dirichlet distribution parameterized by a
  single vector $\mathbf{\alpha}$; this vector, in turn, is computed
  from the polarity scores of discourse desendants, taking into
  account .  We evaluate our proposed method on the IMDB corpus and
  Stanford Sentiment Treebank, getting\todo{add evaluation numbers}.
\end{abstract}

\section{Introduction}\label{intro}

Sentiment analysis has admittedly been one of the most popular,
competitive, and prolific areas of natural language processing since
its very inception around 2004~\cite[pp.~\todo{add pages}]{Liu:12}.
However, most of the research done in this field has been primarily
concentrated either on the analysis of \emph{words}' meaning or
classification of \emph{sentence} polarity and subjectivity, with
little attention paid to semantic links between the sentences---\ie{}
the structure of \emph{discourse}.  Nevertheless, as noted by
\newcite{Polanyi:06}, discourse relations are one of the most
influential factors that might significantly change (and completely
overturn) the semantic orientation of an opinion.

Indeed, if we look at Figure~\ref{fig:rst-tree}, we will see that the
first passage of the cited IMDB review expresses a strong negative
attitude of the author towards any kind of TV shows, and especially
the medical ones.  The second paragraph, however, admits that he/she
liked the series ``House M.D.''  And it is the pragmatic link of
\discrel{Antithesis} that tells us how these two parts relate to each
other, and the higher importance of the second claim that determines
the overall mood of the whole text.

\begin{figure*}[htb!]\label{fig:rst-tree}
  {
  \small
  \centering
  \dirrel{Antithesis}
         {
           \dirrel{}
                  {
                    \dirrel{Background}
                           {\rstsegment{\refr{1}}}
                           {}
                           {\rstsegment{\refr{2}}}}
                  {Cause}
                  {
                    \dirrel{Result}
                           {\rstsegment{\refr{3}}}
                           {}
                           {\rstsegment{\refr{4}}}
                  }
         }
         {}
         {\rstsegment{\refr{4}}}

         \begin{flushleft}
           \begin{rhetoricaltext}
             \unit[1]{I am a physician,}
             \unit[2]{and as an inviolable rule, I \negterm{HATE} medical shows.}
             \unit[3]{Granted, TV series tend to be \negterm{one-dimensional}, due to inherent \negterm{difficulties} in the genre,}
             \unit[4]{but ``doctor shows'' are something I \negterm{avoid} like the
               proverbial \negterm{plague}.}

             \unit[5]{And then one evening I caught ``House, MD'' and was completely \posterm{drawn into} the show.}
             \unit[6]{In House I find the \negterm{anti-hero} that I've been \posterm{waiting for} in a medical show.}
             \unit[7]{The guy who knows everything, but is \negterm{wrong} often enough to keep us all guessing.}
             \rstsource{mndoc, IMBD}
           \end{rhetoricaltext}
         \end{flushleft}
         \caption{An RST tree example for a movie
           review}\label{dasa:fig:rst-tree} }

\end{figure*}

%% \blfootnote{
%%     % % final paper: en-us version
%%     %
%%     % \hspace{-0.65cm}  % space normally used by the marker
%%     % This work is licensed under a Creative Commons
%%     % Attribution 4.0 International License.
%%     % License details:
%%     % \url{http://creativecommons.org/licenses/by/4.0/}.
%% }

\section{Related Work}\label{relwork}

% \done[inline]{\newcite{Bickerstaffe:10}}

% \newcite{Bickerstaffe:10} also considered the rating prediction task,
% addressing this problem with the minimum-spanning-tree (MST) SVM
% approach.  In the initial step of this method, they constructed a
% strongly connected graph whose vertices were associated with the most
% representative example (determined via the average all-pairs Tanimoto
% coefficient) of each star rating and the edge weights represented the
% Tanimoto distances between those nodes.  Afterwards, they determined
% the MST of this graph using the Kruskal's
% algorithm~\cite[see][pp.~567--574]{Cormen:09} and, finally,
% constructed a decision tree from this MST, replacing the MST vertices
% with binary SVM classifiers, which had to discern the respective
% rating groups. An evaluation on the four-star review corpus
% of~\newcite{Pang:05} showed an improvement by up to~7\% over the
% previous state of the art, boosting it to 59.37\% average accuracy.

As it turns out, even the very first works on opinion mining already
pointed out the importance of discourse phenomena for classification
of the overall polarity of a text.  For example, in the seminal paper
of~\newcite{Pang:02}, where the authors tried to predict the semantic
orientation of movie reviews, they quickly realized the fact that it
was insufficient to rely on the mere presence or even the majority of
polarity clues in the text, because these clues could any time be
reversed by a single counter-argument of the critic (see
Example~\ref{disc-snt:exmp-pang02}).  This observation was also
confirmed by \newcite{Polanyi:06}, who ranked discourse relations among
the most important factors that could significantly affect the
intensity and polarity of a sentiment.  To prove this claim, they gave
several convincing examples, where a concessive statement considerably
weakened the strength of a polar opinion, and vice versa, an
elaboration notably increased its persuasiveness.

\newcite{Pang:04} were also among the first who incorporated a
discourse-aware component into a document-level sentiment classifier.
For this purpose, they developed a two-stage system in which the first
predictor distinguished between subjective and objective statements by
constructing a graph of all sentences (linking each sentence to its
neighbors and also connecting it to two abstract polarity nodes) and
then partitioning this graph into two clusters (subjective and
objective) based on its minimum cut; the second classifier then
inferred the overall polarity of the text by only looking at the
sentences from the first (subjective) group.  With this method,
\newcite{Pang:04} achieved a statistically significant improvement
(86.2\% versus 85.2\% for the Na\"{\i}ve Bayes system and 86.15\%
versus 85.45\% for SVM) over classifiers that analyzed all text
sentences at once, without any filtering.
%% (Later on, a similar approach was also proposed by
%% \newcite{Yessenalina:10}~[\citeyear{Yessenalina:10}], who used
%% an expectation-maximization algorithm to select a small subset of
%% the most indicative sentences and then classified the document [as
%% either positive or negative] with the help of this subset,
%% achieving 93.22\% accuracy on the aforementioned IMDB dataset.)

Although an oversimplification, the core idea that locally adjacent
sentences are likely to share the same subjective orientation
(\emph{local coherence}) was dominating the following DASA research
for almost a decade.  For example, \newcite{Riloff:03} also improved the
accuracy of their Na\"{\i}ve Bayes predictor of subjective expressions
by almost two percent after adding a set of local coherence features.
Similarly, \newcite{Hu:04} could better disambiguate users' attitudes to
particular product attributes by taking the semantic orientation of
previous sentences into account.

At the same time, another line of discourse-aware sentiment research
concentrated on the joint classification of all opinions in the text,
where in addition to predicting each sentiment in isolation, the
authors also sought to maximize the ``total happiness'' (\emph{global
  coherence}) of these assignments, ensuring that related subjective
statements received agreeing polarity scores.  Notable works in this
direction were done by \newcite{Snyder:07}, who proposed the Good Grief
algorithm for predicting users' satisfaction with different restaurant
aspects, and \newcite{Somasundaran:08a,Somasundaran:08}, who introduced
the concept of \emph{opinion frames} (OF), a special data structure
for capturing the relations between opinions in discourse.  Depending
on the type of these opinions (arguing~[\emph{A}] or
sentiment~[\emph{S}]), their polarity towards the target
(positive~[\emph{P}] or negative~[\emph{N}]), and semantic
relationship between these targets (alternative~[\emph{Alt}] or the
same~[\emph{same}]), the authors distinguished 32 types of possible
frames (\emph{SPSPsame}, \emph{SPSNsame}, \emph{APAPalt}, etc.),
dividing them into reinforcing and non-reinforcing ones.  In later
works, \newcite{Somasundaran:09a,Somasundaran:09b} also presented two
joint inference frameworks (one based on the iterative classification
and another one relying on integer linear programming) for determining
the best configuration of all frames in text, achieving 77.72\%
accuracy on frame prediction in the AMI meeting
corpus~\cite{Carletta:05}.

%% \done[inline]{\newcite{Somasundaran:09a,Somasundaran:09b}}

%% In a later work, \newcite{Somasundaran:09b,Somasundaran:09a} also
%% introduced a joint inference framework based on the Iterative
%% Classification Algorithm (ICA) and Integer Linear Programming (ILP)
%% for joinly predicting the best configuration of single opinions and
%% their frames.  In this approach, the authors first applied a local SVM
%% classifier to compute the probabilities of polarity classes (positive,
%% negative, or neutral) of individual dialog acts and then harnessed the
%% ICA and ILP systems to determine which of the predicted opinions were
%% connected via opinion frames and whether these frames were reinforcing
%% or not.  Given a perfect information about the opinion links, this
%% joint method outperformed the local classifier by more than 9
%% percentage points, reaching 77.72\% accuracy on the AMI meeting
%% corpus~\cite{Carletta:05}.

%% \done[inline]{\newcite{Mao:06}}

%% \newcite{Mao:06} proposed the idea of isotonic CRFs in which they
%% explicitly modeled the constraint that features which were stronger
%% associated with either polarity classes had to have higher
%% coefficients than less predictive attributes.  After proving that this
%% formalism also allowed to directly model the ordinal scale of
%% sentiment scores (with lower CRF outputs indicating the negativity of
%% a sentence, and higher scores showing its positive class), the authors
%% used this approach to model the sentiment flow in a document.  For
%% this purpose, they first predicted the polarity value for each
%% sentence of a document in isolation and then convolved these outputs
%% with a Gaussian kernel, getting a smoothed polarity curve for the
%% whole analyzed text at the end.
%% \done[inline]{\newcite{Thomas:06}}

%% \newcite{Thomas:06} enhanced an SVM-based sentiment classification
%% system for predicting speaker's attitude in political speeches with
%% information about the inter-speaker agreement, incorporating these
%% links into the global cost function.  Thanks to this change, the
%% authors achieved $\approx$4\% improvement in accuracy (from 66.05 to
%% 70.81\%) over the baseline classifer which analyzed each utterance in
%% isolation.

An attempt to unite local and global coherence was made by
\newcite{McDonald:07}, who tried to simultaneously predict the polarity
of a document and classify semantic orientations of its sentences.
For this purpose, the authors devised an undirected probabilistic
graphical model based on the structured linear
classifier~\cite{Collins:02}.  Similarly to \newcite{Pang:04}, they
connected the label nodes of each sentence to the labels of its
neighboring clauses and also linked these nodes to the overarching
vertex representing the polarity of the text.  After optimizing this
model with the MIRA learning algorithm~\cite{Crammer:03},
\newcite{McDonald:07} achieved an accuracy of 82.2\% for
document-level classification and 62.6\% for sentence-level prediction
on a corpus of online product reviews, outperforming pure document and
sentence classifiers by up to four percent.  A crucial limitation of
this system though was that its optimization required the gold labels
of sentences and documents to be known at the training time, which
considerably limited its applicability to other domains with no such
data.

%% A similar approach was also suggested by~\newcite{Sadamitsu:08}, who
%% attained 82.74\% accuracy on predicting the polarity of customer
%% reviews with the help of hidden conditional random fields.

Another significant drawback of all previous approaches is that they
completely ignored traditional discourse theory and, as a result,
severely oversimplified discourse structure.  Among the first who
tried to overcome this omission were \newcite{Voll:07}, who proposed two
discourse-aware enhancements of their lexicon-based sentiment
calculator (SO-CAL).  In the first method, the authors let the SO-CAL
analyze only the topmost nucleus EDU of each sentence, whereas in the
second approach, they expanded its input to all clauses that another
classifier had considered as relevant to the main topic of the
document.  Unfortunately, the former solution did not work out as well
as expected, yielding 69\% accuracy on the corpus of Epinion
reviews~\cite{Taboada:06}, but the latter system could perform much
better, achieving 73\% on this two-class prediction task.

Other ways of adding discourse information to a sentiment system were
explored by \newcite{Heerschop:11}, who experimented with three
different approaches:
\begin{inparaenum}[(i)]
\item increasing the polarity scores of words that appeared near the
  end of the document,
\item assigning higher weights to nucleus tokens, and finally
\item learning separate scores for nuclei and satellites using a
  genetic algorithm.
\end{inparaenum}
An evaluation of these methods on the movie review corpus
of~\newcite{Pang:04} showed better performance of the first option
(60.8\% accuracy and 0.597 macro-\F), but the authors could
significantly improve the results of the last classifier at the end by
adding an offset to the decision boundary of this method, which
increased both its accuracy and macro-averaged \F{} to 0.72.

Further notable contributions to RST-based sentiment analysis were
made by \newcite{Zhou:11}, who used a set of heuristic rules to infer
polarity shifts of discourse units based on their nuclearity status
and outgoing relation links; \newcite{Zirn:11}, who used a lexicon-based
sentiment system to predict the polarity scores of elementary
discourse units and then enforced consistency of these assignments
over the RST tree with the help of Markov logic constraints; and,
finally, \newcite{Wang:13}, who determined the semantic orientation of a
document by taking a linear combination of the polarity scores of its
EDUs and multiplying these scores with automatically learned
coefficients.

%% \footnote{Similarly to the approach of~\newcite{Zirn:11}, these
%%   coefficients depended on the status of the segment in the RST
%%   tree (whether nucleus or sattelite) and relation, which connected
%%   the respective discourse node to the ancestor.}  A similar system
%%   was also described by \newcite{Chenlo:13,Chenlo:14}, who used their
%%   model to analyze user blog posts, achieving significantly better
%%   results on the TREC corpus \cite{Macdonald:09} than any
%%   discourse-unaware baselines.

Among the most recent advances in RST-aware sentiment research, we
should especially emphasize the work of \newcite{Bhatia:15}, who
proposed two different DASA systems:
\begin{itemize}
\item discourse-depth reweighting (DDR)
\item and rhetorical recursive neural network (R2N2).
\end{itemize}
In the former approach, the authors estimated the relevance
$\lambda_i$ of each elementary discourse unit $i$ as:
\begin{equation*}
  \lambda_i = \max\left(0.5, 1 - d_i/6\right),
\end{equation*}
where $d_i$ stands for the depth of the $i$-th EDU in the document's
discourse tree.  Afterwards, they computed the sentiment score
$\sigma_i$ of that unit by taking the dot product of its binary
feature vector $\mathbf{w}_i$ (token unigrams) with polarity scores
$\boldsymbol{\theta}$ of these unigrams:
\begin{equation*}
  \sigma_i = \boldsymbol{\theta}^{\top}\mathbf{w}_i;
\end{equation*}
and then calculated the overall semantic orientation of the
document~$\Psi$ as the sum of sentiment scores for all units,
multiplying these scores by their respective discourse-depth factors:
\begin{equation*}
  \Psi = \sum_i\lambda_i\boldsymbol{\theta}^T\mathbf{w}_i = \boldsymbol{\theta}^T\sum_i\lambda_i\mathbf{w}_i,
\end{equation*}
In the R2N2 system, the authors largely adopted the RNN method
of~\newcite{Socher:13} by recursively computing the polarity scores of
discourse units as:
\begin{equation*}
  \psi_i = \tanh\left(K_n^{(r_i)} \psi_{n(i)} + K_s^{(r_i)}\psi_{s(i)} \right),
\end{equation*}
where $K_n^{(r_i)}$ and $K_s^{(r_i)}$ stand for the nucleus and
satellite coefficients associated with the rhetorical relation $r_i$,
and $\psi_{n(i)}$ and $\psi_{s(i)}$ represent sentiment scores of the
nucleus and satellite of the $i$-th vertex.  This approach achieved
84.1\% two-class accuracy on the movie review corpus
of~\newcite{Pang:04} and reached 85.6\% on the dataset
of~\newcite{Socher:13}.

For the sake of completeness, we should also note that there exist
discourse-aware sentiment approaches that build upon PDTB and SDRT\@.
For example, \newcite{Trivedi:13} proposed a method based on latent
structural SVM~\cite{Yu:09}, where they represented each sentence as a
vector of features produced by a feature function $\mathbf{f}(y,
\mathbf{x}_i, h_i)$, in which $y\in\{-1, +1\}$ denotes the potential
polarity of the whole document, $h_i \in \{0, 1\}$ stands for the
assumed subjectivity class of sentence $i$, and $\mathbf{x}_i$
represents the surface form of that sentence; and then tried to infer
the most likely semantic orientation of the document $\hat{y}$ over
all possible assignments $\mathbf{h}$, \ie{}:
\begin{equation*}
  \hat{y} =
  \argmax_y\left(\max_{\mathbf{h}}\mathbf{w}^{\top}\mathbf{f}(y,
  \mathbf{x}, \mathbf{h})\right).
\end{equation*}
To ensure that these assignments were still coherent, the authors
additionally extended their feature space with special
\emph{transitional} attributes, which indicated whether two adjacent
sentences were likely to share the same subjectivity given the
discourse connective between them.  With the help of these features,
\newcite{Trivedi:13} could improve the accuracy of the
connector-unaware model on the movie review corpus of~\newcite{Maas:11}
from 88.21 to 91.36\%.

The first step towards an SDRT-based sentiment approach was made by
\newcite{Asher:08}, who presented an annotation scheme and a pilot
corpus of English and French texts that were analyzed according to the
SDRT theory and enriched with additional sentiment information.
Specifically, the authors asked the annotators to ascribe one of four
opinion categories (reporting, judgment, advice, or sentiment) along
with their subclasses (\eg{} inform, assert, blame, recommend) to each
discourse unit that had at least one opinionated word from a sentiment
lexicon.  Afterwards, they showed that with a simple set of rules, one
could easily propagate opinions through SDRT graphs, increasing the
strengths or reversing the polarity of the sentiments, depending on
the type of the discourse relation that was linking two segments.

In general, however, PDTB- and SDRT-based sentiment systems are much
less common than RST-inspired solutions.  Because of this fact, we
will primarily concentrate on the RST-based of methods.  In
particular, for the sake of comparison, we replicated the linear
combination approach of \newcite{Wang:13} and also reimplemented the
DDR and R2N2 systems of~\newcite{Bhatia:15}.  Furthermore, to see how
these techniques would perform in comparison with much simpler
baselines, we additionally created two methods that predicted the
polarity of a message by only considering its last or topmost nucleus
EDU (henceforth \textsc{Last} and \textsc{Root}), and also estimated
the results of our original LBA classifier without any
discourse-related modifications (henceforth \textsc{No-Discourse}).

Apart from the above baselines and existing methods, we propose
several novel DASA solutions, which will be briefly described below.


\section{Recursive Dirichlet Model}\label{method}


\section{Evaluation}\label{sec:evaluation}

\subsection{Data}\label{subsec:eval:data}
\subsection{Baselines}\label{subsec:eval:data}

\begin{table}[hbt]
  \begin{center}
    \begin{tabular}{p{0.3\columnwidth}%
        *{2}{>{\centering\arraybackslash}p{0.3\columnwidth}}} % first columm
      \toprule
      \multirow{2}*{\bfseries Method} & %
      \multicolumn{2}{c}{\bfseries Dataset}\\\cmidrule(lr){2-3}
      & IMDB & SST\\\midrule
      RD & &\\
      R2N2~\cite{Bhatia:15} & 84.1 & 85.6\\
      DDR~\cite{Bhatia:15} & 82.9 & 82.0\\
      \bottomrule
    \end{tabular}
    \caption{Accuracy of sentiment classification on IMDB~\cite{Pang:04}
      and Stanford Sentiment
      Treebank~\cite{Socher:13}}\label{tbl:accuracy}
  \end{center}
\end{table}

\subsection{Conclusion}\label{conclusion}

% include your own bib file like this:
\bibliographystyle{coling2020/acl}
\bibliography{bibliography}
\end{document}
